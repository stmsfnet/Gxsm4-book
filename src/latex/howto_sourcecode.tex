%/* Gxsm-4.0 - Gnome X Scanning Microscopy
% * universal controlling and data analysis software for STM and AFM: Documentation
% * 
% * This software is based on the earlier documentation of XSM, GXSM, GXSM2, GXSM3
% * 
% * Copyright (C) 2023 Percy Zahl, Thorsten Wagner, Andreas Klust, Stefan Schröder
% *
% * Authors: Percy Zahl <zahl@users.sf.net>, Thorsten Wagner <stm@users.sf.net>
% * WWW Home: http://gxsm.sf.net
% *
% * This program is free software: you can redistribute it and/or modify
% * it under the terms of the GNU General Public License as published by
% * the Free Software Foundation, either version 3 of the License, or
% * (at your option) any later version.
% *    
% * This program is distributed in the hope that it will be useful,
% * but WITHOUT ANY WARRANTY; without even the implied warranty of
% * MERCHANTABILITY or FITNESS FOR A PARTICULAR PURPOSE.  See the
% * GNU General Public License for more details.
% *   
% * You should have received a copy of the GNU General Public License
% * along with this program.  If not, see <http://www.gnu.org/licenses/>.
% */

\chapter{HowTo: Install \Gxsm\ from source code}

\section{System requirements}
\Gxsm\ needs a reasonably up-to-date machine running a recent version of almost any Linux variant. \Gxsm\ is developed on a Debian Testing. Some description to install \Gxsm\ on a clean Debian bullseye are found in this forum posting: \url{https://sourceforge.net/p/gxsm/discussion/297458/thread/3f0faafe/} It also runs on Ubuntu 22.04 LTS for which dedicated binary packages are available. 

System memory requirements are ranging from little to several gigabytes if you want to deal with big scans, movies or multidimensional data sets.

To compile \Gxsm\ on a Debian Testing install these devel packages:

\begin{verbatim}
$ apt-get install libgnomeui-dev intltool \
yelp-tools gtk-doc-tools gnome-common \ 
libgail-3-dev libnetcdf-dev \
libnetcdf-cxx-legacy-dev libfftw3-dev \
libgtk-4-0 libgtksourceview-5.0-dev \
gsettings-desktop-schemas-dev \ 
python-gobject-2-dev libgtksourceviewmm-5.0-dev \
libquicktime-dev libglew-dev freeglut3-dev \ 
libgl1-mesa-dev libopencv-core-dev \ 
libopencv-features2d-dev libopencv-highgui-dev \
libopencv-objdetect-dev libnlopt-dev libglm-dev \
fonts-freefont-ttf meson ninja
\end{verbatim}

The `\textbackslash' will connect the command lines. You can also skip it an write everything in one long command line.
 
\subsection{Wayland, NVIDIA, openGL4...}

The 3D visualization requires an openGL4 support based on a NVIDIA GPU (and the nouveau driver). \Gxsm\ will run without 3D visualization. If 3D is selected without in the channelselector, possible a lot of warnings will pop up. This will prevent \Gxsm\ from crashing but the warning dialog windows make \Gxsm\ effectively unresponsive.

To run GXSM4 with Wayland as window manager, you have two alternative to tweak your linux: i) In Ubuntu 22.04 Wayland is the default window manager if you are not using an nvidia gpu. To deactivate Wayland support, please add/enalbe as root in /etc/gdm3/custom.conf the line.

\begin{verbatim}
WaylandEnable=false
\end{verbatim}

ii) Alternatively, deactivate the splash screen during GXSM4 startup. Open the dconf-editor and navigate to org/gnome/gxsm4. Here change the entry "splash" to off.

\section{Source code}

Recently, the source code from \url{https://sf.net} to \url{https://github.com}. To obtain a copy of the source code, please run in a terminal:

\begin{verbatim}
$ cd ~
$ git clone https://github.com/pyzahl/Gxsm4 gxsm4-git
\end{verbatim}

The repository will be cloned into ~/gxsm4-git.

\subsection{Meson}
Since \Gxsm\ the meson build system (instead of make) is used. After download the source code (into ~/gxsm4-git) change into the respective directory from a terminal and run the following commands:

\begin{verbatim}
$ meson builddir
$ cd builddir
$ meson compile
$ meson install
\end{verbatim}

First command creates the `builddir' in the project's root folder. Here, `\$meson compile, meson install' calls ninja. You can simply call `ninja install' to do it all. You may have to run the last line with a preceding `sudo' to gain the rights to copy files in system directories. 

To uninstall call in the `buildir'

\begin{verbatim}
$ ninja uninstall
\end{verbatim}

%%% Local Variables: 
%%% mode: latex
%%% TeX-master: "Gxsm-main"
%%% End: 

