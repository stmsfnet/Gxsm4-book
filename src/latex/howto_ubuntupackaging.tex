%/* Gxsm-4.0 - Gnome X Scanning Microscopy
% * universal controlling and data analysis software for STM and AFM: Documentation
% * 
% * This software is based on the earlier documentation of XSM, GXSM, GXSM2, GXSM3
% * 
% * Copyright (C) 2023 Percy Zahl, Thorsten Wagner, Andreas Klust, Stefan Schröder
% *
% * Authors: Percy Zahl <zahl@users.sf.net>, Thorsten Wagner <stm@users.sf.net>
% * WWW Home: http://gxsm.sf.net
% *
% * This program is free software: you can redistribute it and/or modify
% * it under the terms of the GNU General Public License as published by
% * the Free Software Foundation, either version 3 of the License, or
% * (at your option) any later version.
% *    
% * This program is distributed in the hope that it will be useful,
% * but WITHOUT ANY WARRANTY; without even the implied warranty of
% * MERCHANTABILITY or FITNESS FOR A PARTICULAR PURPOSE.  See the
% * GNU General Public License for more details.
% *   
% * You should have received a copy of the GNU General Public License
% * along with this program.  If not, see <http://www.gnu.org/licenses/>.
% */

\chapter{HowTo: Make \Gxsm\ Ubuntu packages}

\section{System requirements}

Make sure that you have all required packages installed. After downloading the source you might have a look at the respective control-file stores in the subfolder debian:

\begin{verbatim}
$ apt-get install libgnomeui-dev intltool \
yelp-tools gtk-doc-tools gnome-common \ 
libgail-3-dev libnetcdf-dev \
libnetcdf-cxx-legacy-dev libfftw3-dev \
libgtk-4-0 libgtksourceview-5.0-dev \
gsettings-desktop-schemas-dev \ 
python-gobject-2-dev libgtksourceviewmm-5.0-dev \
libquicktime-dev libglew-dev freeglut3-dev \ 
libgl1-mesa-dev libopencv-core-dev \ 
libopencv-features2d-dev libopencv-highgui-dev \
libopencv-objdetect-dev libnlopt-dev libglm-dev \
fonts-freefont-ttf meson ninja \
debhelper dh-autoreconf dh-make devscripts
\end{verbatim}

The `\textbackslash' will connect the command lines. You can also skip it an write everything in one long command line.
 
\section{Source code}

Recently, the source code from \url{https://sf.net} to \url{https://github.com}. To obtain a copy of the source code, please run in a terminal:

\begin{verbatim}
$ cd ~
$ git clone https://github.com/pyzahl/Gxsm4 gxsm4-git
\end{verbatim}

The repository will be cloned into ~/gxsm4-git.

\subsection{Packing}

The source code also contains all files to make your own deb-package. To do so, checkout the source from \url{https://github.com} as described above. Then enter the folder ~/gxsm4-git and the subfolder debian (assuming that you checked out the source code in gxsm4-git) and rename control\_<your ubuntu> to control.

Then you can use 

\begin{verbatim}
$ dch -i
$ dch -r
\end{verbatim}

to update the changelog file of the package. In particular, you might want to increase the version number. Then use

\begin{verbatim}
$ debuild -b -I -uc -us
\end{verbatim}

to make your (unsigned) package. To make a signed package use

\begin{verbatim}
$ debuild -b -I -k<your email>
\end{verbatim}

The last command will require the variables DEBEMAIL and DEBFULLNAME to be set, i.e., you may want to add to your .bashrc\\

\begin{verbatim}
DEBEMAIL="<your email>"
DEBFULLNAME="<your full name>"
export DEBEMAIL DEBFULLNAME
\end{verbatim}

\section{Installation}

Package built this way can be installed via 
\begin{verbatim}
$ sudo dpkg -i <package-name>
\end{verbatim}

\section{Starting \Gxsm}

You can start \Gxsm via the terminal by using `gxsm4' or via the activity panel.

\section{Ubuntu binaries}
An easy alternative to compiling the source code is to obtain binaries from \url{https://launchpad.net}. Right now, the repository just hosts binaries of GXSM3 for Ubuntu (up to 22.04).

\subsection{Adding the package repository}
First, you have to add the PPA (Personal Package Archive) hosted on \url{https://launchpad.net} and refresh the local database:

\begin{verbatim}
$ apt-add-repositry ppa:totto/gxsm
$ apt update
\end{verbatim}

Please, follow the instructions in the terminal.

\subsection{Package installation}
You can install \Gxsm\ via

\begin{verbatim}
$ apt install gxsm4 
\end{verbatim}

This command will not only install \Gxsm\ but also makes sure that the required dependencies are installed. 

For SRanger MK2 or MK3 based hardware support, do not forget to install the respective kernel modules via

\begin{verbatim}
	$ apt install sranger-modules-std-dkms sranger-modules-mk23-dkms
\end{verbatim}

Once you have installed \Gxsm\ you will be noticed by your package manager (in the lastest Ubuntu version this is called Ubuntu Software Center) about updates.

NOTE: In any case, please do not mix up an installation based on the source code and one via APT. The installation via meson usually will use the folder `/usr/local/', whereas apt will put \Gxsm\ into the folder `/usr/'. The flatpak installation is done in a kind of sandbox so that it does not interfere with other ways of installation.


%%% Local Variables: 
%%% mode: latex
%%% TeX-master: "Gxsm-main"
%%% End: 

